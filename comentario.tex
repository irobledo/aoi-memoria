\section{Conclusiones.}

Resumiendo todo el contenido desarrollado en este artículo, parece claro que la reutilización y el reciclaje de los dispositivos electrónicos y eléctricos puede resultar muy beneficioso para el ser humano y especialmente para el medio ambiente.

Pero la realidad es bien diferente, a las compañías actualmente no parece interesarles demasiado, suponiéndoles un coste que no están dispuestos a afrontar. Los gobiernos están legislando pero de forma muy suave. Han tratado de favorerecer la adopción de estas medidas con la introducción de tasas y subvenciones que ayuden a financiar y hacer económicamente soportables estos procesos. Al amparo de estas tasas y subvenciones han surgido multitud de empresas organizadas para realizar esta tarea.

Sin embargo, como se ha dejado constancia, algunas de estas compañías simplemente están aprovechando las subvenciones y los impuestos para enriquecerse aún más si llegar a prestar el servicio que deben. 

Occidente sigue utilizando su posición predominante para aprovecharse de los distintos países en desarrollo o en vías de desarrollo, utilizándolos como sus vertederos a larga distancia.

Sin embargo, existe un lado positivo, y es la cada vez mayor conciencia del consumidor final sobre el tratamiento que debe darse a la e-basura, la amplia cobertura que este tema empieza a tener en los medios de comunicación y la difusión que empieza a tener y la aparición de compañías y grupos de experto que buscan como desarrollar efectivamente este proceso.

Aún estamos lejos de lo mencionado en \cite{reusing-silicon} pero debemos ser positivos y pensar que estamos en el camino. Sólo hace falta un empujón de la sociedad y una toma de conciencia definitiva de los políticos de los países desarrollados para desarrollar una potente legislación y unos mecanismos de vigilancia y sanción adecuados.